\documentclass[t]{beamer}
\usepackage[brazil]{babel}
\usepackage[T1]{fontenc}
\usepackage{amsmath}
\usepackage{amssymb}
\usepackage{amsfonts}
\usepackage{amsthm}
\usepackage{graphicx}
\usepackage{xcolor}
%\usepackage[scaled]{helvet}
\usepackage[sfdefault]{FiraSans}
\renewcommand{\familydefault}{\sfdefault}

\newcommand{\thetitle}{Extração de dados processuais dos Tribunais de Justiça para auxílio ao Jornalismo Investigativo}
\newcommand{\theauthor}{João Paulo Taylor Ienczak Zanette}

\title{\thetitle}
\author{\theauthor}

%%%
%%% Define cores
%%%
\definecolor{cinza}{HTML}{75818B}

%%%
%%% Remove a barra de navegação do Beamer
%%%
\setbeamertemplate{navigation symbols}{}

%%%
%%% Margem dos slides
%%%
\setbeamersize{text margin left=10mm,text margin right=5mm}

%%%
%%% Redefine a fonte do título dos slides
%%%
\setbeamercolor{frametitle}{fg=cinza}
\setbeamerfont{frametitle}{series=\bfseries}
\setbeamerfont{frametitle}{size=\huge}

%%%
%%% Ajusta a posição do título dos slides e início do texto
%%%
\addtobeamertemplate{frametitle}{\vspace*{2mm}}{\vspace*{5mm}}

%%%
%%% Adiciona páginação nos slides
%%%
%%% Caso não queira, basta comentar este bloco inteiro
%%% para ocultar a paginação
%%%
\addtobeamertemplate{navigation symbols}{}{
\usebeamerfont{footline}
\usebeamercolor[fg]{footline}
}
\setbeamercolor{footline}{fg=cinza}
\setbeamerfont{footline}{series=\bfseries}
\setbeamerfont{footline}{size=\tiny}
\setbeamertemplate{footline}{
\usebeamerfont{page number in head}
\usebeamercolor[fg]{page number in head}
\hspace{5mm}
\insertframenumber/\inserttotalframenumber
\vspace{5mm}
}

%%%
%%% Redefine símbolo padrão do itemize
%%%
\setbeamertemplate{itemize items}[ball]

%%%
%%% Insere numeração nas figuras
%%%
\setbeamertemplate{caption}[numbered]

%%%
%%% Imagem de fundo a ser usada em todos os slides (exceto
%%% no primeiro e no último)
%%%
\usebackgroundtemplate
{
\includegraphics[width=\paperwidth,height=\paperheight]{fundo.png}
}

%%%
%%% Adiciona slide de "Estrutura"
%%%
\AtBeginSection[]{\frame{\frametitle{Estrutura}\tableofcontents
[current]}}

%%%
%%% Define fontes e cores do slide de "Estrutura"
%%%
\setbeamerfont{section in toc}{series=\bfseries}
\setbeamercolor{section in toc}{fg=gray}
\setbeamerfont{section in toc shaded}{series=\mdseries}
\setbeamercolor{section in toc shaded}{fg=gray!01}
\setbeamercolor{subsection in toc}{fg=cinza}
\setbeamercolor{subsection in toc shaded}{fg=gray!60}
\setbeamercolor{subsubsection in toc}{fg=cinza}
\setbeamercolor{subsubsection in toc shaded}{fg=gray!60}

\mode<presentation>
